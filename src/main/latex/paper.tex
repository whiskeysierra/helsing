\documentclass[a4paper]{article}
\usepackage[utf8]{inputenc}
\usepackage{fullpage}
\usepackage{csquotes}
\usepackage[ngerman]{babel}
\usepackage{biblatex}
\usepackage{float}
\usepackage{graphicx}
\usepackage{subfigure}
\usepackage[format=plain,labelfont=bf,up]{caption}
\usepackage{hyperref}
\bibliography{paper.bib}
\title{Tenzing \\ A SQL Implementation On The MapReduce Framework}
\author{Willi Schönborn}
\date{\today}
\begin{document}

\begin{figure}[H]
\centering
\includegraphics{beuth.png}
\maketitle
\end{figure}

\section*{Einleitung}
Als Teil der Lehrveranstaltung \textit{Programmierung - Fortgeschrittene Konzepte} im Wintersemester 2011/2012 an der \textit{Beuth Hochschule für Technik Berlin} sollte im Rahmen einer Semesterarbeit ein wissenschaftliches Paper ausgewählt, untersucht und bewertet werden. Das Ziel dieses Dokumentes ist es die Ergebnisse dieser Semesterarbeit zusammenzufassen.

Ausgewählt wurde das Google-Paper \textit{Tenzing A SQL Implementation On The MapReduce Framework} \cite{TENZING}. Erschienen ist das Paper als Teil der Proceedings zur 37th VLDB, der \textit{International Conference on Very Large Data Bases} im September 2011. Die Autoren sind die Google-Mitarbeiter Biswapesh Chattopadhyay, Liang Lin, Weiran Liu, Sagar Mittal, Prathyusha Aragonda, Vera Lychagina, Younghee Kwon und Michael Wong.

\section*{Kurzbeschreibung}
Tenzing beschreibt eine SQL92-kompatible Implementierung auf Basis des Google-eigenen MapReduce-Frameworks \cite{MAPREDUCE}.

Ausgewählt wurde das Paper 

\newpage

\nocite{GFS}
\nocite{GOOGLE-TENZING}
\nocite{GOOGLE-MAPREDUCE}
\nocite{GOOGLE-GFS}
\nocite{WP-SQL92}
\nocite{WP-SQL99}
\printbibliography

\end{document}

